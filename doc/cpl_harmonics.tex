\documentclass{article}


\usepackage{amsmath, amsthm, amssymb, amsfonts}
\usepackage[makeroom]{cancel}
\usepackage{xcolor}
\usepackage{graphicx}
\usepackage{geometry}
\usepackage[american, RPvoltages]{circuitikz}

\usepackage{IEEEtrantools}

\usepackage{hyperref}
\usepackage[utf8]{inputenc}
\usepackage[english]{babel}

\geometry{
    textheight=9in,
    textwidth=6.5in,
    top=1in,
    headheight=12pt,
    headsep=25pt,
    footskip=30pt
}

\newcommand{\smallsignal}[2]{\lowercase{#1}_{\lowercase{#2}}}
\newcommand{\equationname}{Equation}

\newcommand{\cancelRed}[1]{\textcolor{red}{\cancel{\textcolor{black}{#1}}}}
\newcommand{\cancelToOne}[1]{\textcolor{red}{\cancelto{1}{\textcolor{black}{#1}}}}
\newcommand{\cancelToZero}[1]{\textcolor{red}{\cancelto{0}{\textcolor{black}{#1}}}}

\begin{document}

% ------------------------------------------------------------------------------
% Cover Page and ToC
% ------------------------------------------------------------------------------

\title{\textbf{\uppercase{Constant Power Load Harmonics}}}
\date{}
\author{\textbf{Eric Ponce} \\
		Massachusetts Institute of Technology}

\maketitle

\tableofcontents

% ------------------------------------------------------------------------------

\section{Introduction}

This paper demonstrates the inherent presence of current harmonics in constant power loads driven by voltage pertubations.
It also presents a solution for the fourier series of the current, and compares it to the standard small signal model.
The circuit may be described as follows:

\begin{figure}[htbp]
\center
\begin{circuitikz}
	\draw (2, 0) node[ground]{};
	\draw (2, 0) -- ++(-2, 0)
		to[vsource, v=$V_T$] ++(0, 2)
		to[vsourcesin, v=$v_t$] ++(0, 2) -- ++(4, 0)
		to[generic] ++(0, -4) -- ++(-2, 0);
\end{circuitikz}
\caption{Constant Power Load Test Circuit.}
\end{figure}

Consider a constant power load for power $P$ being driven by a voltage $v_T(t) = V_T + v_t\cos{t}$ where $V_I$ is the bias voltage and $v_i$ is the pertubation amplitude.
The current for this circuit is dictated by the relation $P = v_T(t) i_T(t)$ and can be found as:

\begin{equation}
i_T = \frac{P}{v_T(t)} = \frac{P}{V_T + v_t \cos{t}}
\end{equation}

This \emph{is} a nonlinear function and contain harmonics at multiples of the fundamental.

\newpage
\section{Fourier Series}

To solve for the fourier coefficients of the constant power load current, we start with a reminder of the definition of the fourier coefficients of a function $s(x)$ with a period of $2\pi$:

\begin{equation}
A_0 = \frac{1}{2\pi} \int_{-\pi}^{\pi} s(x) dx
\end{equation}

\begin{equation}
A_n = \frac{1}{\pi} \int_{-\pi}^{\pi} s(x) \cos(nx) dx
\end{equation}

\begin{equation}
B_n= \frac{1}{\pi} \int_{-\pi}^{\pi} s(x) \sin(nx) dx
\end{equation}

Since the function is even, the $B_n$ coefficients will be zero. 
To begin to solve for the rest of the coefficients, we will rewrite the current into a more useful exponential form:

\begin{equation}
i_T = \frac{P}{v_T(t)} = \frac{P}{V_T + v_t \cos{t}} = \frac{P}{V_T + \frac{v_t}{2}(e^{jt}+e^{-jt})} = \frac{2P}{2V_T + v_t(e^{jt}+e^{-jt})}
\end{equation}

\subsection{DC Coefficient}

We can compute $A_0$ as follows using a substitution of $z=e^{jt}$ to convert it into a contour integral, followed by the use of the residue theorem around one of the poles:

\begin{IEEEeqnarray}{rCl}
A_0 &=& \frac{1}{2\pi} \int_{-\pi}^{\pi} \frac{2P}{2V_T + v_t(e^{jt}+e^{-jt})} dt \nonumber\\
	&=& \frac{P}{j\pi} \oint \frac{1}{z} \frac{1}{2V_T + v_t(z+z^{-1})} dz \nonumber\\
	&=& \frac{P}{jv_t\pi} \oint \frac{1}{z^2 + \frac{2V_T}{v_t}z + 1} dz \nonumber\\
	&=& \frac{P}{jv_t\pi} 2\pi j \mathop{\mathrm{Res}}_{z=z_0} \left(\frac{1}{z^2 + \frac{2V_T}{v_t}z + 1}\right) \nonumber\\
	&=& \frac{2P}{v_t} \mathop{\mathrm{Res}}_{z=z_0} \left(\frac{1}{(z-z_0)(z-z_1)}\right).
\end{IEEEeqnarray}

where $z_0$ and $z_1$ are the poles of the function and can be found using \eqref{eq:poles}.
The residue for this function is found by taking the limit:

\begin{IEEEeqnarray}{rCl}
\mathop{\mathrm{Res}}_{z=z_0} \left(\frac{1}{(z-z_0)(z-z_1)}\right) &=& \lim_{z\rightarrow z_0} \left(\frac{z-z_0}{(z-z_0)(z-z_1)}\right) \nonumber\\
	&=& \lim_{z\rightarrow z_0} \left(\frac{1}{(z-z_1)}\right) \nonumber\\
	&=& \frac{1}{(z_0-z_1)}
\end{IEEEeqnarray}

Finally, we arrive at the coefficient using \eqref{eq:pole_identity1}:

\begin{equation}
A_0 = \frac{2P}{v_t} \mathop{\mathrm{Res}}_{z=z_0} \left(\frac{1}{z^2 + \frac{2V_T}{v_t}z + 1}\right) = \frac{2P}{v_t} \frac{1}{2\sqrt{\left(\frac{V_T}{v_t}\right)^2 - 1}} = \frac{P}{\sqrt{V_T^2 - v_t^2}}.
\end{equation}

\subsection{Harmonic Coefficients}

We can compute $A_n$ as follows using a substitution of $z=e^{jt}$ and the same definitions of $z_0$ and $z_1$ as above:

\begin{IEEEeqnarray}{rCl}
A_n &=& \frac{1}{\pi} \int_{-\pi}^{\pi} \frac{2P \frac{e^{jnt}+e^{-jnt}}{2}}{2V_T + v_t(e^{jt}+e^{-jt})} dt \nonumber\\
	&=& \frac{P}{j\pi} \oint \frac{1}{z} \frac{z^n+z^{-n}}{2V_T + v_t(z+z^{-1})} dz \nonumber\\
	&=& \frac{P}{j\pi v_t} \oint \frac{1}{z^n}\frac{1+z^{2n}}{(z-z_0)(z-z_1)} dz \nonumber\\
	&=& \frac{2P}{v_t} \left(\mathop{\mathrm{Res}}_{z=z_0} \left(\frac{1}{z^n}\frac{1+z^{2n}}{(z-z_0)(z-z_1)}\right) + \mathop{\mathrm{Res}}_{z=0} \left(\frac{1}{z^n}\frac{1+z^{2n}}{(z-z_0)(z-z_1)}\right) \right).
\end{IEEEeqnarray}

Let's start with the residue at $z=z_0$:

\begin{equation}
\mathop{\mathrm{Res}}_{z=z_0} \left(\frac{1}{z^n}\frac{1+z^{2n}}{(z-z_0)(z-z_1)}\right) = \lim_{z\rightarrow z_0} \left(\frac{1+z^{2n}}{z^n (z-z_1)}\right) = \frac{1+z_0^{2n}}{z_0^n (z_0-z_1)}
\end{equation}

For the the residue at $z=0$, we use \eqref{eq:pole_identity3} and \eqref{eq:residue_identity1}:

\begin{IEEEeqnarray}{rCl}
	\mathop{\mathrm{Res}}_{z=0} \left(\frac{1}{z^n}\frac{1+z^{2n}}{(z-z_0)(z-z_1)}\right) &=& \mathop{\mathrm{Res}}_{z=0} \left(\frac{1+z^{2n}}{z^n}\frac{1}{z_0 - z_1}\sum_{k=0}^\infty z^k\left(z_0^{k+1} - z_1^{k+1}\right)\right) \nonumber\\
	&=& \lim_{z\rightarrow z_0} \left(\frac{1}{z^n}\frac{z}{z_0 - z_1}\left.\left(\sum_{k=0}^\infty z^k\left(z_0^{k+1} - z_1^{k+1}\right)\right)\right\vert_{k=n-1}\right) \nonumber\\
	&=& \lim_{z\rightarrow z_0} \left(\frac{1}{z^n}\frac{z}{z_0 - z_1}\left(z^{n-1} \left(z_0^{n} - z_1^{n}\right)\right)\right) \nonumber\\
	&=& \lim_{z\rightarrow z_0} \left(\frac{1}{z_0 - z_1}\left(z_0^{n} - z_1^{n}\right)\right) \nonumber\\
	&=& \frac{\left(z_0^{n} - z_1^{n}\right)}{z_0 - z_1}
\end{IEEEeqnarray}

Finally, use \eqref{eq:pole_identity1} and \eqref{eq:pole_identity2} for the total overall solution:

\begin{IEEEeqnarray}{rCl}
	A_n &=& \frac{2P}{v_t} \left(\mathop{\mathrm{Res}}_{z=z_0} \left(\frac{1}{z^n}\frac{1+z^{2n}}{(z-z_0)(z-z_1)}\right) + \mathop{\mathrm{Res}}_{z=0} \left(\frac{1}{z^n}\frac{1+z^{2n}}{(z-z_0)(z-z_1)}\right) \right) \nonumber\\
	&=& \frac{2P}{v_t} \left(\frac{1+z_0^{2n}}{z_0^n (z_0-z_1)} + \frac{\left(z_0^{n} - z_1^{n}\right)}{z_0 - z_1} \right) \nonumber\\
	&=& \frac{2P}{v_t(z_0-z_1)} \left(z_0^{n}+z_0^{-n} + z_0^{n} - z_1^{n} \right) \nonumber\\
	&=& \frac{2P}{v_t(z_0-z_1)} \left(2 z_0^{n}+\cancelRed{z_0^{-n} - z_0^{-n}} \right) \nonumber\\
	&=& \frac{4Pz_0^n}{v_t(z_0-z_1)} \nonumber\\
	&=& \frac{2P}{z_1^n\sqrt{V_T^2-v_t^2}}.
\end{IEEEeqnarray}

\subsection{Overall Fourier Series}

Combining the DC and harmonic terms, we arrive at the final solution:

\begin{IEEEeqnarray}{rCl}
	\label{eq:fourier}
	i_T(t) = \frac{P}{\sqrt{V_T^2 - v_t^2}} + \sum_{n=1}^{\infty} \frac{2P}{z_1^n\sqrt{V_T^2-v_t^2}} \cos{nt}
\end{IEEEeqnarray}

If $V_T >> v_t$


\begin{IEEEeqnarray}{rCl}
	z_1 &\approx& -\frac{2V_T}{v_t}\\
	i_T(t) &\approx& \frac{P}{V_T} + \sum_{n=1}^{\infty} \frac{2P}{z_1^nV_T} \cos{nt} \nonumber\\
	&=& \frac{P}{V_T} + \sum_{n=1}^{\infty} \frac{P}{2^{n-1}V_T^{n+1}} v_t \cos{nt} \label{eq:approx_fourier}
\end{IEEEeqnarray}

\section{Small Signal Approximation}

A first order taylor expansion on $i_T(v_T)$ around $v=V_T$ results in

\begin{IEEEeqnarray}{rCl}
i_T(v_T) &\approx& i_T(V_T) + \left. \frac{\partial i_T}{\partial v_T} \right\vert_{v=V_T} (v_T-V_T) \nonumber\\
	&=& I_T + \left. \frac{\partial i_T}{\partial v_T} \right\vert_{v=V_T} (v_T-V_T) \nonumber\\
	&=& \frac{P}{V_T} - \frac{P}{V_T^2} v_t \cos{t}. \label{eq:small_signal}
\end{IEEEeqnarray}

\section{Conclusion}

We have presented the solution for the fourier series of the current drawn by a constant power load in the presense of a voltage pertubation in \eqref{eq:fourier} and its approximate form under small-signal conditions in \eqref{eq:approx_fourier}.
Also, we have presented the traditional small signal approximation of the current in \eqref{eq:small_signal}.

Crucially, the first two terms of \eqref{eq:approx_fourier} and \eqref{eq:small_signal} and so while the small signal approximation is overall accurate in fundamental magnitude, it \emph{neglects} harmonics introduced by the non-linear constant power load.
These harmonics play a role in relfected impedance across a mixing interface, such as a passive rectifier.

\setcounter{section}{0}
\renewcommand\thesection{\Alph{section}}
\renewcommand\theequation{\Alph{section}.\arabic{equation}}
\section*{Appendix}

\setcounter{equation}{0}
\section{Two Real Pole Functions}

Let $f(z) = z^2 + 2A z + 1$, where $A > 0$, it's poles would be
\begin{IEEEeqnarray}{rCl}
	z_0 &=& -A + \sqrt{A^2 - 1}, \quad \mathrm{and} \\
	z_1 &=& -A - \sqrt{A^2 - 1}. \label{eq:poles}
\end{IEEEeqnarray}

These poles may combined into useful identities:
\begin{IEEEeqnarray}{rCl}
	z_0-z_1 &=& 2\sqrt{A^2 - 1} \label{eq:pole_identity1}, \\
	z_0 z_1 &=& 1 \label{eq:pole_identity2}.
\end{IEEEeqnarray}

Using these identities, and the geometric series of $1/(1-a) = \sum_{k=0}^\infty a^k$, $f(z)$ may be rearranged as follows:
\begin{IEEEeqnarray}{rCl}
f(z)&=& \frac{1}{(z-z_0)(z-z_1)} \nonumber\\
	&=& \cancelToOne{\frac{1}{z_0 z_1}} \frac{1}{\left(1-\frac{z}{z_0}\right)\left(1-\frac{z}{z_1}\right)} \nonumber\\
	&=& \frac{1}{z_0 - z_1}\left(\frac{z_0}{1-\frac{z}{z_1}} - \frac{z_1}{1-\frac{z}{z_0}}\right) \nonumber\\
	&=& \frac{1}{z_0 - z_1}\left(z_0\sum_{k=0}^\infty\left(\frac{z}{z_1}\right)^k - z_1\sum_{k=0}^\infty\left(\frac{z}{z_0}\right)^k\right) \nonumber\\
	&=& \frac{1}{z_0 - z_1}\sum_{k=0}^\infty z^k\left(z_0\left(\frac{1}{z_1}\right)^k - z_1\left(\frac{1}{z_0}\right)^k\right) \nonumber\\
	&=& \frac{1}{z_0 - z_1}\sum_{k=0}^\infty z^k\left(z_0^{k+1} - z_1^{k+1}\right). \label{eq:pole_identity3}
\end{IEEEeqnarray}

\setcounter{equation}{0}
\section{Contour Integrals}

\begin{equation}
\oint z^k dz = 	2 \pi j \mathop{\mathrm{Res}}_{z=0}(z^k) = \begin{cases}
					2\pi j \quad &\text{if} \, k=-1 \\
					0 \quad &\text{otherwise}
				\end{cases}. \label{eq:residue_identity1}
\end{equation}

Using the above, we can create a even more general identity:
\begin{IEEEeqnarray}{rCl}
\oint \sum_{k}^{\infty} a^k z^k dz &=& \sum_{k}^{\infty} \oint  a^kz^k \,dz	\nonumber\\
	&=& \cancelToZero{\sum_{k=-2}^{-\infty}\oint a^{k}z^{k} \,dz} + \oint a^{-1}z^{-1} \,dz + \cancelToZero{\sum_{k=0}^{\infty} \oint a^{k}z^{k} \,dz} \nonumber\\
	&=& \frac{2\pi j}{a} \label{eq:residue_identity2}
\end{IEEEeqnarray}




% \subsection{Pictures}

% \begin{figure}[htbp]
%     \center
%     \includegraphics[scale=0.06]{img/photo.jpg}
%     \caption{Sydney, NSW}
% \end{figure}

% \subsection{Citation}

% This is a citation\cite{Eg}.

\newpage

% ------------------------------------------------------------------------------
% Reference and Cited Works
% ------------------------------------------------------------------------------

\bibliographystyle{IEEEtran}
% \bibliography{References.bib}

% ------------------------------------------------------------------------------

\end{document}